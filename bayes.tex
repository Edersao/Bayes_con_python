\documentclass[a4paper,10pt]{article}
\usepackage[utf8]{inputenc}
\usepackage{enumerate}

\title{Bayes con Python}
\author{Edersao}

\begin{document}
\section{ Introducción a Bayes}
¿Crees que llueva? Es una pregunta cotidiana, en general, el comenzar cualquier cosa con ¿Crees...? es algo muy inherente de nuestro día a día: ¿Crees que llueva? ¿Crees que le guste? ¿Crees que pase el examen? etc. Muchas de estas preguntas surgen por que desconocemos algún aspecto del evento en que estamos pensando u observando. Por ejemplo, retomando la primera pregunta ¿Crees que llueva? por nuestra experiencia y observación de nuestro entorno podemos emitir una respuesta a base de nuestra creencia (pueda o no ocurrir), por lo que podemos  asignar algún  significado cualitativo para responder este tipo de preguntas. En otras palabras que me gustan, estamos dando un significado a nuestra incertidumbre.

Este tipo de preguntas,  no solo ¿Crees? sí no cualquier otra que se refiera a un futuro próximo de cualquier observación, puede ayudarse de la probabilidad. 
Por lo tanto, estamos definiendo a la probabilidad como una medida de la incertidumbre de una observación X.
 
\end{document}